\documentclass[4apaper, 12pt]{article}
\usepackage[utf8]{inputenc}
\usepackage{amsmath}
\usepackage{amssymb}
\usepackage{xcolor}
\title{Las ecuaciones que cambiaron el mundo}
\author{Adrian Archundia }
\date{October 2022}
\usepackage{fancyhdr}
\pagestyle{fancy}
\fancyhf{}
\lfoot{\thepage}
\rhead{Adrian Archundia}
\lhead{\leftmark}

\begin{document}

\maketitle

\section{Fisica}
\begin{itemize}
    \item [*] La aceleración de un objeto es directamente proporcional a la fuerza que actúa sobre él e inversamente proporcional a la masa
    \begin{equation*}
        F=ma
    \end{equation*}
    \item[$\sharp$]La velocidad se define como el movimiento en el cual un objeto se desplaza en línea recta, en una sola dirección, recorriendo distancias iguales en el mismo intervalo de tiempo, manteniendo en todo su movimiento una velocidad constante y sin aceleración.
    $$V=\frac{d}{t}$$
    
   \item[$\clubsuit$] La ley de la gravitación universal formulada por Isaac Newton postula que la fuerza que ejerce una partícula puntual con masa $m_1$ sobre otra con masa $m_2$ es directamente proporcional al producto de las masas, e inversamente proporcional al cuadrado de la distancia que las separa
    \textcolor{red}{\[F=G\frac{m_1m_2}{r^2}\]}
    
    \item[$\spadesuit$]Para cada punto del espacio-tiempo, la ecuación del campo de Einstein describe cómo el espacio-tiempo se curva por la materia y tiene la forma de una igualdad local entre un tensor de curvatura para el punto y un tensor que describe la distribución de materia alrededor del punto. Esta es la ecuación "general" incluyendo la constante cosmológica, según Einstein "el mayor error de su carrera".
    \begin{equation*}
        R_{\mu\nu}-\frac{1}{2}g_{\mu\nu} R+g_{\mu\nu}\varLambda=\frac{8\pi G}{c^4}T_{\mu\nu}
    \end{equation*}
    
    \item[$\mho$]El principio de incertidumbre de Heisenberg nos dice que para varias copias idénticas de un sistema en un estado determinado, las medidas de la posición y de la cantidad de movimiento variarán de acuerdo con una cierta distribución de probabilidad característica del estado cuántico del sistema.
    
    $$\sigma_x \sigma_p\geq \frac{\hbar}{2} $$
    
   \item [*] La ecuación de Schrödinger describe la evolución temporal de una partícula masiva. Es de importancia central en la teoría de la mecánica cuántica.
    
    \begin{equation*}
        i\hbar\frac{\partial}{\partial t}\varPsi(r,t)=-\frac{\hbar^2}{2m}\triangledown^2\varPsi(r,t)+V(r)\varPsi(r,t)
    \end{equation*}
    
   \item[$\sharp$] Es la ecuación que describe la dilatación temporal que sufre un cuerpo al viajar a velocidades relativistas.
    
    \[t=\frac{T}{\sqrt{1-\frac{v^2}{c^2}}}\]
    
    \item[$\clubsuit$] Estas son las cuatro ecuaciones de Maxwell en el vacio, sin cargas presentes. Estas ecuaciones junto con la fuerza de Lorentz son las que explican cualquier tipo de fenómeno electromagnético.
    
    $$\triangledown\cdot E=0$$
    $$\triangledown\cdot B=0$$
    $$\triangledown\times E=-\frac{\partial B}{\partial t}$$
    $$\triangledown\times B=\frac{1}{c^2}\frac{\partial E}{\partial t}$$
    
    \item[$\spadesuit$] La famosa ecuación de la teoría de la relatividad de Einstein que relaciona masa y energía, a través de las constante c, velocidad de la luz.
    \begin{equation*}
        E=mc^2
    \end{equation*}
     \item[$\mho$]La ley de Coulomb señala que la fuerza F con que dos carga eléctricas $q_1$ y $q_2$ se atraen o repelen es proporcional al producto de las mismas e inversamente proporcional al cuadrado de la distancia r que las separa.
    \begin{equation*}
        F=K\frac{q_1q_2}{r^2}
    \end{equation*}
    \item[$\spadesuit$] La “distribución normal”, una ecuación empleada tanto en biología como en física para modelar propiedades. Por ejemplo, describe el comportamiento de grandes grupos de procesos independientes.
      \begin{equation*}
          f(x,\mu,\sigma)=\frac{1}{\sigma\sqrt{2\Pi}}e^{-\frac{x-\mu}{2\sigma^2}}
      \end{equation*}
      \item[$\mho$] la “ecuación de onda” que no es sino una ecuación diferencial que describe cómo una propiedad está cambiando a través del tiempo en términos de derivado de esa propiedad; esto es, describe la propagación de una variedad de ondas, como las ondas sonoras, las ondas de luz y las ondas en el agua, lo que ayudó enormemente en los campos como el electromagnetismo, la acústica o la dinámica de fluidos, unificando fenómenos tan dispares como la luz, el sonido o los terremotos.
      \[\frac{\partial u}{\partial t^2}=c^2\frac{\partial u}{\partial x^2}\]
    \item [*]la “transformada de Fourier”. Esta ecuación que los expertos consideran imprescindible para la comprensión de las estructuras de onda más complejas como puede ser el propio lenguaje humano (esencial en el tratamiento de señales).
    $$g(\xi)=\frac{1}{\sqrt{2\pi}}\int_{-\infty}^{\infty}f(x)e^{-i\xi x}\mathrm{d}x$$
    \item[$\sharp$]Las “ecuaciones de Navier-Stokes”. Claude-Louis Henri Navier y George Gabriel Stokes describieron esta ecuación en 1845 para explicar la mecánica de fluidos, con increíbles implicaciones en el mundo de la ingeniería. Es la base de la aerodinámica y la hidrodinámica. 
    \begin{equation*}
        \rho(\frac{\partial v}{\partial t}+v\cdot\triangledown v)=-\triangledown p+\triangledown\cdot T+f
    \end{equation*}
    \section{Matematicas}
    \item [*] El teorema de Pitagoras describe la relación entre los lados de un triángulo rectángulo en una superficie plana, conceptos esenciales para la comprensión de la geometría. Gracias a él se conectó el álgebra y la geometría.
    $$a^2+b^2=c^2$$
    \item[$\sharp$] Gracias a los logaritmos y hasta el desarrollo de los ordenadores, esta base de cálculo fue la más rápida para multiplicar grandes cantidades ya que permitió simplificar operaciones muy complejas.
    \begin{equation*}
        \log_b(x\cdot y)=\log_b(x)+\log_b(y)
    \end{equation*}
     \item[$\clubsuit$]la “fórmula de la definición de la derivada en cálculo”. Esta ecuación ayudó a comprender el cambio de las funciones cuando sus variables cambiaban.
     \begin{equation*}
         \textcolor{red}{\frac{df}{dt}=\lim_{h\rightarrow0}\frac{f(t+h)-f(t)}{h}}
     \end{equation*}
      
    \item[$\clubsuit$] Una fórmula general, en la definición más amplia del término, es aquella que, en el ámbito de las matemáticas, permite obtener el valor de una incógnita en distintos casos particulares.
    \begin{equation*}
        \textcolor{blue}{x=-b\pm\frac{\sqrt{b^2-4ac}}{2a}}
    \end{equation*}
    \item[$\spadesuit$]La identidad de Euler está considerada como la más bella de las ecuaciones, ya que pone en escena una combinación improbable de cinco constantes matemáticas.
    \begin{equation*}
        F-E+V=2
    \end{equation*}
    \item[$\mho$] El teorema de Tales es una ley de la geometría que nos indica que si se traza una línea paralela a cualquiera de los lados de un triángulo tendremos como resultado un triángulo semejante el triángulo original.
    \begin{equation*}
        \frac{AB}{AD}=\frac{AE}{AC}=\frac{DE}{BC}
    \end{equation*}
    \section{Quimica}
    \item [*] La variación de la concentración de iones hidronio ($H_3O^+$) puede oscilar entre 1 M (molar) y 10-14, ambos números son muy incómodos para expresar la acidez o basicidad de una solución, por lo tanto se trabaja con una escala logarítmica, que lleva por nombre pH.
    \begin{equation*}
       \textcolor{brown}{pH=-\log_{10}[H_3O^+]}
    \end{equation*}
      \item[$\sharp$]La concentración molar, es una medida de la concentración de un soluto en una disolución, ya sea alguna especie molecular, iónica o atómica.
      \[M=\frac{n}{L}\]
     \item[$\clubsuit$] La normalidad, que se representa con la letra N, es una unidad de concentración química que expresa el número de equivalentes de un soluto que hay por cada litro de solución.
     $$N=\frac{\#eq}{v}$$
      \item[$\spadesuit$] La entalpía específica (h) de una sustancia es su entalpía por unidad de masa. Es igual a la entalpía total (H) dividida por la masa total (m). Tenga en cuenta que la entalpía es la cantidad termodinámica equivalente al contenido total de calor de un sistema.
      $$H=E+PV$$
      \item[$\mho$] El concepto de gas ideal es una extrapolación del comportamiento de los gases reales a densidades y presiones bajas hacia el comportamiento ideal.
      $$PV=nRT$$
\end{itemize}


\end{document}
